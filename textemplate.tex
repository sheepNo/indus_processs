\documentclass[11pt]{article}

\usepackage[french]{babel}
\usepackage[utf8]{inputenc}
\usepackage[T1]{fontenc}
\usepackage{eurosym}

% Use the postscript times font!
\usepackage{times}

\usepackage{listings}
\usepackage{geometry}

\usepackage{graphicx}
\usepackage{caption}
\usepackage{subcaption}

\usepackage{amsmath}
\usepackage{amssymb}
\usepackage{amsfonts}
\usepackage{amsthm}
\usepackage{algorithm}
\usepackage{algorithmicx}
\usepackage{algpseudocode}

\newtheorem{theorem}{Theorem}
\newtheorem{lemma}{Lemma}

\newcommand\underrel[2]{\mathrel{\mathop{#2}\limits_{#1}}}

\geometry{hmargin=2.0cm, vmargin=2.0cm}

%%%%%%%%%%%%%%%%%%%%%%%%%%%%%%%%%%%%%%%%%%%%%%%%%%%%%%%%%%%%%%%%%%%%%%%%%%%%%%%%%%%%%%%%
% Title, authors and addresses

\title{\textbf{Industrial processes for high-confidence design}\\ TP SystemC}
\date{\today}
\author{Etienne Bontemps \and Andrey Sosnin}

%%%%%%%%%%%%%%%%%%%%%%%%%%%%%%%%%%%%%%%%%%%%%%%%%%%%%%%%%%%%%%%%%%%%%%%%%%%%%%%%%%%%%%%%
\begin{document}

\maketitle

\section*{Exercice 0}
\begin{enumerate}
    \item \texttt{sc\_module} et \texttt{sc\_channel}: Elles représentent les composants physiques du système (opérationnels ou communiquants).
    \item \texttt{SC\_METHOD} et \texttt{SC\_THREAD}. À la différence du process qui s'exécute en entier, le thread peut être arrêté. Les deux peuvent se suspendre.
    \item \texttt{wait} et \texttt{notify} peuvent être utilisés pour contrôler les threads. Ces méthodes sont utilisées comme des mutex, des barrières (variables conditionnelles) ou des sémaphores.
\end{enumerate}

\section*{Exercice 1}
\begin{enumerate}
    \item Ajout de quelques lignes dans consumer...
    \item 180ns pour \texttt{./fifo\_perf 1} 100ns pour \texttt{./fifo\_perf 10000} (décroissant).
\end{enumerate}

\section*{Exercice 2}
\begin{enumerate}
    \item La syntaxe de sensitive est moche. (la sensitivité doit suivre la déclaration du tread)
    \item ok
    \item   \begin{itemize}
                \item Tout fonctionne sans réécrire la source.
                \item \texttt{No target at this address 0x250}
                \item \texttt{No target at this address 0xffffffff}
                \item \texttt{Segmentation Fault}. le simulateur lui-même a segfault à l'exécution ; il n'a pas \emph{simulé} de segfault.
            \end{itemize}
    \item Utilisation de \texttt{sc\_time\_stamp} avec des dans les \texttt{read} et \texttt{write} \texttt{wait(double, SC\_NS)}
    2240ns vs 2880ns (2 64bytes transfers)

    il faut retirer la macro dans le Makefile
\end{enumerate}


\end{document}
